%---------------------------------------------
% This document is for pdflatex
%---------------------------------------------
\documentclass[11pt]{article}

\usepackage{amsmath,amsfonts,amssymb,graphicx,natbib,setspace,authblk}
\usepackage{float}
\usepackage[running]{lineno}
\usepackage[vmargin=1in,hmargin=1in]{geometry}

\usepackage{enumitem}
\setlist{topsep=.125em,itemsep=-0.15em,leftmargin=0.75cm}

\usepackage{gensymb}

\usepackage{longtable}
\setlength{\LTcapwidth}{\linewidth}


\usepackage[compact]{titlesec} 

\usepackage{bm,mathrsfs}

\usepackage{ifpdf}
\ifpdf
\DeclareGraphicsExtensions{.pdf,.png,.jpg}
\usepackage{epstopdf}
\else
\DeclareGraphicsExtensions{.eps}
\fi

\graphicspath{{../figures/}}

\renewcommand{\floatpagefraction}{0.98}
\renewcommand{\topfraction}{0.99}
\renewcommand{\textfraction}{0.05}

\clubpenalty = 10000
\widowpenalty = 10000

%%%%%%%%%%%%%%%%%%%%%%%%%%%%%%%%%%%%%%%%%%%%% 
%%% Just for commenting
%%%%%%%%%%%%%%%%%%%%%%%%%%%%%%%%%%%%%%%%%%%%
\usepackage[usenames]{color}
\newcommand{\new}{\textcolor{red}}
\newcommand{\comment}{\textcolor{blue}}

\newcommand{\be}{\begin{equation}}
\newcommand{\ee}{\end{equation}}
\newcommand{\ba}{\begin{equation} \begin{aligned}}
\newcommand{\ea}{\end{aligned} \end{equation}}

\def\X{\mathbf{X}}

\floatstyle{boxed}
\newfloat{Box}{tbph}{box}

\title{Can historical data predict population responses to climate change experiments?}

\author[a]{Andrew R. Kleinhesselink\thanks{Corresponding author. Department of Wildland Resources and the Ecology Center, Utah State University, Logan, Utah Email: arklein@aggiemail.usu.edu}}
\author[a]{Peter B. Adler}
\author[a]{Andrew Tredennick}
\affil[a]{Department of Wildland Resources and the Ecology Center, Utah State University, Logan, Utah}


\renewcommand\Authands{ and }

\date{Last compile: \today} 

\sloppy

\renewcommand{\baselinestretch}{1.25}

\begin{document}

\maketitle

\textbf{\large{Keywords:}} Climate change, demographic models, rain-out shelter. 

\bigskip \textbf{Running title:} Predicting climate response

\smallskip \textbf{Article type:} Letter

\smallskip \textbf{Authorship statement:} ARK and PBA designed the experiment and supervised data collection. AT helped with the data analysis.
\smallskip 

\setlength{\parindent}{8ex}

\newpage

\begin{doublespacing} 

\linenumbers

\section*{Summary}

\begin{enumerate}
\item Climate is an important driver of population ecology however there have been few tests of whether observational data that links population performance with climate variation can be used to predict the responses of populations to experimentally imposed climate conditions.   
\item Using longterm historical observational data from a sagebrush steppe plant community, previous research has reported a wide variety of climate effects on four dominant plant species. We tested whether statistical models based on these observational data could be used to predict how these species respond to a five year drought and irrigation experiment conducted from 2011 to 2016. 
\item We established sixteen new plots at the same field cite as the original historical data and assigned each to either drought or irrigation treatments imposed using rainfall shelters and automatic sprinklers. The original plots were also monitored and served as ambient climate controls. In order to describe the demographic effects of the climate manipulations, we used all data, observational and experimental, to fit a set of statistical models describing each species survival, growth and recruitment in response to local competitive interactions and the effects of the climate manipulation experiment. Next we fit another set of models to only the observational data that used seasonal soil moisture covariates to describe annual variation in the vital rates.  Finally, we used the observation-based models to predict the effects of the experimental treatments on each species vital rates. We also used an individual based population model to predict one year ahead changes in population size in the experimental plots.   
\item The experimental drought and irrigation treatments successfully lowered and increased soil moisture respectively.  Over the course of the experiment, average plot cover of \textit{Hesperostipa comata} and \textit{Pseudoroegneria spicata} declined significantly, while cover of \textit{Poa secunda} showed a trend towards increase in the irrigated plots. At the level of individual vital rates, experimental drought reduced growth and survival of \textit{Hesperostipa comata} and \textit{Poa secunda} and the survival of \textit{P. spicata}, while drought increased the growth of small \textit{A. tripartita}. 
\item Soil moisture parameters improved model predictions for six out of the twelve models we examined.  Moreover, we observed strong positive correlations between the predicted treatment effects and the treatment effects we observed.   
\item At the population-level, including the effects of soil moisture improved the cover predictions made by individual based models for \textit{P. secunda} and \textit{P. spicata}.   
\item \emph{Synthesis}: Observational climate data did hold valuable information for predicting species' responses to this climate change experiment. We were encouraged that treatment responses were often in the right direction to predicted responses even when the effect were not signficant. We generally were better able to predict species responses to the drought treatment than to the control or the irrigation treatment. This suggests that soil moisture is an important factor in predicting the population dynamics of these species but only when water is truly limiting.
  

\end{enumerate}


\section*{Introduction}

Climate is widely considered one of the most powerful external forces driving changes in species abundance across space and time \citep{teller_linking_2016}.  The effects of climate on populations and ecosystems are most apparent at the largest scales in space and time: climate determines the distribution of ecosystems, treelines and the the range limits of many species \citep{}, while the recent historical and the paleoecological record shows that long-term climate change leads to changes in species range limits \citep{parmesan, davis}. Understanding and predicting the effects of climate on populations is an increasingly important goal if we are to understand and predict the effects of climate change on earth's ecosystems. 

Unfortunately, for many species it is difficult to determine how annual climate variation affects populations and demographic rates \citep{teller_linking_2016, ehrlen_advancing_2016}. Observational data on species performance across years with varying climate can provide some information on how climate might affect population abundance \citep{dalgleish_climate_2010,ehrlen_advancing_2016}. However, many years of data are needed to reliably detect climate effects, especially when annual variation in demographic rates is high \citep{teller_linking_2016,gerber_optimal_2015}. Generating predictions from observational historical data for the future novel conditions created by climate change is fraught with risk. Climate change will not only change mean annual temperature and precipitation, but also affect annual variation in these measures, and possibly the covariance between them \citep{}. Models based on the historical response of populations to annual climate variation, will therefore be extrapolating beyond the range of observed climate variation when they are used to generate predictions in the future \citep{}. If models fit to historical data can be used to accurately predict the effects of experimental climate manipulations, especially manipulations that generate extreme variation in climate, it would be strong confirmation that the climate effects they describe are real and will hold in the future \cite{adler_can_the_past_2014}.  

Many plant species occur across a wide range of climates \cite{} and individual plants in many environments must tolerate large fluctuations in seasonal temperature and soil moisture.   Unlike most animals, plants must endure these conditions in place and often become dormant in less optimal seasons \textit{}. At first pass, these observation would seem to indicate that annual variation in climate should have little effect on plant population performance. 

Nevertheless, there is abundant evidence that plant performance shows high year to year variation, both at the level of individual growth, survival and reproduction and in terms of total population abundancend often this variation c. In many regions interannual variation in precipitation can be directly linked to variation in net primary productivity \citep{knapp and smith, Hsu and Adler}. The growth rates of trees are also often tightly linked to annual precipitation, so much so that annual growth rings in their stems can serve as accurate records of historical climate thousands of years before the present \citep{yang}.  Likewise, many smaller shrub and sub-shrub species show strong variation in growth in response to climate that is recorded in their tissues \citep{Franklin, Srur}. Among annual plants in desert ecosystems, germination and reproductive output is tightly linked to interannual precipitation variation \citep{venables}.


Linking annual climate variation to demographic performance in plants is a high priority for building population models for plants that can predict their future response to climate change \citep{crone_ability_2013,ehrlen_advancing_2016}. However, despite the clear signs that climate drives net primary productivity at the ecosystem level and the variation in individual growth rate in trees and other woody species, as well as the germination and fecundity of annual plants, there are few studies that have connected the effects of interannual climate variation to population models for plants, and fewer still that have tested whether these population models can be used to accurately predict the future responses of plants to short term climate variation. Adler et al. \citep{adler byrn and leiker} showed that population models based on historically observed correlations between plant population growth rates and precipitation did have some predictive power in describing species response to a short-term climate manipulation in a North American grassland. Three species showed responses to experimentally imposed drought and irrigation that were well predicted by population models fitted to historical observations.  However, another three species, showed responses to the experimental conditions that were not well predicted by historical observations.  The authors suggested that limited replication in the historical data for two of these species and changing competitive conditions in the community may have led to the poor predictions.  

The demography and competitive interactions between three dominant grass species and a dominant shrub species at in a sagebrush steppe plant community at the US Sheep Experiment Station near Dubois, Idaho have been described in at least X different studies since 2005. X of these studies report significant effects of historical climate variation on the vital rates and overall population growth of these species. Both precipitation and temperature have been shown to have strong species-specific effects on this system.  Although past studies used different statistical models and methods for variable selection they indicate X.  This well-studied system offers the ideal opportunity to test whether statistical associations between annual climate variation and plant demography in long-term observational data can be used to predict the future responses of plant populations to climate change.  

In this study, we report how the four dominant plant species at the USSES responded to a five year drought and irrigation experiment and use the results to address two research questions: first, how much do the growth, recruitment and survival of our target species differ between the precipitation manipulation treatments? If our experiment does affect species vital rates we interpret that as strong evidence that variation in precipitation should have an effect on populations. Second, we test whether statistical models parameterized from observational data only can accurately predict each species response to the experimental precipitation manipulation? If models based on observational effectively capture the effects of climate on species performance, they should also predict the effects of precipitation treatments in the experiment. 

\section*{Methods}

\subsection*{Study site and data set description}

The U.S. Sheep Experiment Station (USSES) is located 9.6 km north of Dubois, Idaho (44.2\degree N, 112.1\degree W), 1500 m above sea level. During the period of data collection (1926 – 1957), mean annual precipitation was 270 mm and mean temperatures ranged from -8\degree C (January) to 21\degree C (July). The vegetation is dominated by a shrub, \textit{Artemisia tripartita}, and three perennial C3 grasses: \textit{Pseudoroegneria spicata}, \textit{Hesperostipa comata}, and \textit{Poa secunda}. These dominant species account for over 70\% of basal cover and 60\% of canopy cover at this site. 

Scientists at the USSES established 26 1-m$^2$ quadrats between 1926 and 1932. Eighteen quadrats were distributed among four ungrazed exclosures, and eight were distributed in two paddocks grazed at medium intensity spring through fall. All quadrats were located on similar topography and soils. In most years until 1957, all individual plants in each quadrat were mapped using a pantograph (Blaisdell 1958). The historical data set is public and available online \citep{zachmann_mapped_2010}. In 2007, we located 14 of the original quadrats, all of which are inside permanent livestock exclosures, and resumed annual mapped censusing using the traditional pantograph method. Daily temperature and precipitation has been monitored throughout this period at a climate station located at the USSES headquarters (station id: GHCND:USC00102707) which located within 2 km of the research plots.  We downloaded daily and monthly tmin, tmax, and precipitation data from the National Climate Data Centers online database.  

We extracted data on survival, growth, and recruitment from the mapped quadrats based on plants' spatial locations. Our approach tracks genets representing individual plants.  For the shrub, each genet is associated with the basal position of a stem.  For the bunchgrasses, each genet represents a spatially distinct polygon in the mapped quadrat. These genets may fragment and/or coalesce over time. Each mapped polygon is classified as a surviving genet or a new recruit based on its spatial location relative to genets present in previous years \citep{lauenroth_demography_2008}. We modeled vital rates using data from 21 year-to-year transitions between 1929 and 1957, and four year-to-year transitions from 2007 to 2011.  

\subsection*{Precipitation experiment}
In spring 2011, we selected locations for an additional 16 quadrats for the precipitation experiment. We located these in a large exclosure containing six of the historical permanent quadrats.  We avoided plots falling on hill slopes, areas with greater than 20\% bare rock, or with over 10\% cover of the woody shrubs \textit{Purshia tridentata} or \textit{Amelanchier utahensis}. New plots were established in pairs, and one plot per pair was randomly assigned to either the precipitation reduction or the precipitation addition treatment. We mapped the quadrats in June, 2011 and then built the rainfall shelters and set-up the irrigation systems in the fall of 2011. We used a rain-out shelter and automatic irrigation design described in \citep{gherardi, yahdjian_rainout_2002}. Each rain-out shelter covered an area of 2.5 by 2 m and consisted of transparent acrylic shingles held up over the plot to channel 50\% of incoming rainfall off of the plot and into 75 l reservoirs. The collected water was pumped out of reservoirs and sprayed onto paired irrigation treatment plots. Pumping was initiated automatically with float switches that were triggered when water levels in the reservoirs were approximately 20 l, or equivalently irrigation was triggered once for every 6 mm of rainfall collected. We disconnected the irrigation pumps in late fall each year and re-connected them in April.  The drought shelters remained in place throughout the year.  

We monitored soil moisture and air temperature in four of the precipitation experiment plot pairs using Decagon Devices (Pulman Washington) 5TM and EC-5 soil moisture sensors and 5TE temperature sensors.  We installed two soil moisture sensors in each monitored plot at 5 cm and two at 25 cm deep in the soil.  Air temperature was measured underneath the roofing of the shelter at 30 cm above ground. For each pair of manipulated plots, we also installed sensors in a nearby area to measure ambient rainfall and temperature. Data were logged automatically every four hours. We augmented automatic monitoring of the climate in these plots with by taking direct measurements of soil moisture with a handheld EC-5 soil moisture sensor at six points around each plot on 6/6/2012, 4/29/2015, 5/7/2015, 6/9/2015 and 5/10/2016. We analyzed these spot measurements for significant treatment effects on soil moisture using a linear mixed effects model with the \textit{lmer} package in \textit{R} \citep{bates}, with plot, plot group, and date as random effects in the model.    

We conducted a simple statistical to determine the net effect of the experimental treatments on cover in the experiment. First we calculated the log change in cover for each of the four focal species in each quadrat from from the start of the experiment in spring prior to manipulation, to the last year of the experiment. Log change in cover was defined as , $log(Cover_{2016}/Cover_{2011})$ where $Cover_{2016}$ is the cover of each species in 2011 and $Cover_{2015}$ is cover in 2011. We tested for the effect of precipitation treatment on this measure with a linear model in \textit{R}.

\subsection*{Soil moisture modeling}

We expected that our precipitation manipulation experiment would affect plants by altering available soil moisture during the growing season.  Because we do not have direct soil moisture measures for each year of observed plant cover in the historical record, we used the SOILWAT soil moisture model to estimate daily soil moisture at the USSES from 1925 to the present \citep{Parton 1978}. We used an enhanced version of soilwat that has recently been developed for use in semi-arid shrubland ecosystems \citep{Bradford}. SOILWAT uses daily weather data, ecosystem specific vegetation properties and site specific soil properties to estimate water balance processes. SOILWAT specifically estimates rainfall interception by vegetation, evaporation of intercepted water, snow melt and snow redistribution, infiltration into the soil, percolation through the soil, bare-soil evaporation, transpiration from each soil layer, and drainage. We parameterized SOILWAT with the generic sagebrush steppe vegetation parameters and site specific soil texture and bulk density data. We used daily weather data collected at the USSES from 1925 until the present as weather forcing data for the SOILWAT predictions.  

We averaged daily soil moisture predictions from SOILWAT from upper 40 cm of soil and then averaged these seasonally to serve as the covariates in the vital rate regressions for each species. Because we did not monitor soil moisture directly in all control, drought and irrigation plots, we used a model to describe the average treatment effects on soil moisture during the course of the experiment. To do this we first averaged observed soil moisture data by date and plot and then standardized these by the mean and standard deviation of the control soil moisture conditions observed within each plot group. We then found the difference between the soil moisture in the treated plots and the ambient conditions. We then modeled these treatment effects as a function of season and whether a day was rainy or dry. We expected that our drought and irrigation treatments might be more effective during rainy weather than during dry weather. Rainy days were defined as any days when any precipitation was recorded and average temperatures were above 3 degrees C. The day immediately following rainfall was also classified as rainy. We fit this model using the \textit{lmer} package in \textit{R} \citep{bates} with random effects for plot group and date. We then used this model to predict the treatment effects on soil moisture for the entire study period from the ambient soil moisture values predicted from the SOILWAT model described above. These adjusted soil moisture values reflected the average season and rainfall dependent effects of the experimental treatments on soil moisture and could be used as covariates for predicting the effects of our manipulation on each species demographic rates. 


\subsection*{Statistical models of vital rates}

For each of the vital rates and each species we fit three separate models.  First we fit a treatment model fit to all observations in the historical data as well as the contemporary experiment.  This model included parameters estimating the effects of the drought and irrigation treatments on each species vital rate. This model was used to describe the basic results of the experiment. Next we fit two models to the historical observational data only (including the first four years of the modern data 2007 to 2010). We used these two models to generate predictions for the effects of the experiment. In the first prediction model, year to year variation in vital rates was treated as random effect. In the second prediction model, the "climate" model, we included paramaters that described the effects of year to year variation in seasonal soil moisture. These three models allow for two meaningful comparisons.  First we can compare the predictions made by the two prediction models directly to the raw data in the experimental plots.  In addition, we can compare the direction and magnitude of the soil moisture parameters estimated from the observational data by the climate model to the coefficients describing the treatment effects in the treatment model.      

All three versions of the models of the models above follow the same basic structure and are developed from previous work \citep{adler_coexistence_2010,chu_large_2015}. We model the survival probability of an individual genet as a function of genet size, the neighborhood-scale crowding experienced by the genet from both conspecific and heterospecific genets, temporal variation among years, and permanent spatial variation among groups of quadrats (`group'; here means a set of nearby quadrats located within one pasture or grazing exclosure). In this analysis we only include crowding from the four main focal species. 

Formally, we modeled the survival probability, $S$, of genet $i$ in species $j$, group $g$, and removal treatmnt $h$, from time $t$ to $t+1$  as
\begin{equation}
\mbox{logit}(S_{ijgh,t}) = \varphi_{jg}^S + \gamma_{j,t}^S  + \beta_{j,t}^S u_{ij,t} +  
\left \langle \boldsymbol{\omega}_{j,t}^S, \boldsymbol{W}_{ij,t} \right \rangle 
\label{eqn:survReg}
\end{equation}
where $\varphi$ is the spatial group dependent intercept, $\gamma$ is a year-effect, $\beta$ is year-dependent coefficient that represents the effect of log genet size, $u$, on survival in year $t$. $\boldsymbol{\omega}$ is a vector of interaction coefficients which determine the impact of crowding, $\boldsymbol{W}$, by each species on the focal species. The vector $\boldsymbol{W}$ includes crowding from the four dominant species, \textit{A. tripartita}, \textit{P. spicata}, \textit{H. comata}, and \textit{Poa secunda}. 
$\left \langle \boldsymbol{x, y} \right \rangle$ denotes the inner product of vectors $\boldsymbol{x}$ and $\boldsymbol{y}$, 
calculated as \texttt{sum(x*y)} in R.

To describe the treatment effects in the experiment a new term is added to the above model,
\begin{equation}
 \boldsymbol{T}\chi_{j}^S
\label{eqn:survT}
\end{equation}
where $\chi$ is a vector of treatment effect coefficients describing the effects of each experimental treatment $h$ on the survival rate, and $\boldsymbol{T}$ is a design matrix indicating the treatment level of each observation in the data. The design matrix also includes terms for the interaction between plant size $u$ and the treatment effects. These interaction terms allow the effect of each treatment to vary with plant size.  

In the climate model, the $\chi$ and $\xi$ terms are replaced with, 
\begin{equation}
\boldsymbol{C}\xi_{j}^S 
\label{eqn:survC}
\end{equation}
where $\xi$ gives a vector of coefficients describing the effects of a set of soil moisture covariates $\boldsymbol{C}$ in treatment $h$ and year $t$ on the survival rate of species $j$. $\boldsymbol{C}$ is a vector of seasonal average soil moisture and can include interaction effects between plant size, $u$, and the soil moisture covariates that allow the effects of soil moisture to vary with plant size. 

Our growth model has a similar structure. The change in genet size from time $t$ to $t+1$ , conditional on survival, is given by:
\begin{equation}
u_{ijgh,t+1} = \varphi_{jg}^G + \gamma_{j,t}^G + \chi_{jh}^G  + \beta_{j,t}^G u_{ij,t} + 
\left \langle  \boldsymbol{\omega}_{j,t}^G, \boldsymbol{W}_{ij,t} \right \rangle + \varepsilon_{ij,t}^G .
\label{eqn:growReg}
\end{equation}
To capture non-constant error variance in growth, we modeled the variance $\varepsilon$  about the growth curve (\ref{eqn:growReg})  as a nonlinear function of predicted genet size:
\begin{equation}
Var(\varepsilon_{ij,t}^G) = a \exp ^{(bu_{ij,t+1})} .
\label{eqn:growVar}
\end{equation}

As in the survival regression above, specific specific parameters describing the treatment effects on growth are added in the treatment model, 
\begin{equation}
\boldsymbol{T}\chi_{j}^G 
\label{eqn:growT}
\end{equation}
where $\chi$ is a treatment effect describing the effect of experimental treatment $h$ on growth, including treatment by size interactions.

Similarly, in the climate model soil moisture influences the growth equation through these terms,  
\begin{equation}
\boldsymbol{C}\xi_{j}^G 
\label{eqn:growC}
\end{equation}
where $\xi$ is a vector of coefficients describing the effects of soil moisture covariates in the matrix $\boldsymbol{C}$ for treatment $h$ and year $t$ on growth of species $j$. Again this can include interactions between soil moisture and plant size $u$.

Although the main focus of the current analysis the effects of soil moisture, we also modeled the effects of inter- and intra-specific competition in our vital rate models.  We model the crowding experienced by a focal genet as a function of the distance to and size of neighbor genets. In previous work, we assumed that the decay of crowding with neighbor distance followed a Gaussian function \citep{chu_large_2015}, but here we use a data-driven approach \citep{teller_linking_2016, adler_in_prep}. We model the crowding experienced by genet $i$ of species $j$ from neighbors of species $m$ as the sum of neighbor areas across a set of concentric annuli, $k$, centered at the plant,
\begin{equation}
w_{ijm,k} = F_{jm}(d_{k})A_{i,k}     
\label{eqn:wik}
\end{equation}
where $F_{jm}$ is the competition kernel (described below) for effects of species $m$ on species $j$, 
$d_{k}$ is the average of the inner and outer radii of annulus $k$, 
and $A_{im,k}$ is the total area of genets of species $m$ in annulus $k$ around genet $i$. The total crowding on 
genet $i$ exerted by species $m$ is
\begin{equation}
W_{ijm}  =\sum_k {w_{ijm,k}} .
\label{eqn:wijm}
\end{equation} 
Note that $W_{ijj}$ gives intraspecific crowding. The $W$'s are then the components of the $\boldsymbol{W}$ vectors introduced as covariates in the survival (\ref{eqn:survReg}) and growth (\ref{eqn:growReg}) regressions.

We assume that competition kernels $F_{jm}(d)$ are non-negative and decreasing, so that distant plants have less effect than close plants. Otherwise, we let the data dictate the shape of the kernel by fitting a spline model 
using the methods of Teller et al. (2016). The shape of $F_{jm}$ is determined by a set of spline basis coefficients $\vec{b}_{jm}$
and a smoothing parameter $\eta$ that controls the complexity of the fitted kernel. 
Demographic models such as \eqref{eqn:survReg} then have $\gamma$, $\varphi$, $\chi$ , 
$\beta$, $\boldsymbol{\omega}$, $\vec{b}$ and $\eta$ as parameters to be fitted. We implemented this in the statistical computing environment, \texttt{R}, 
by making the spline coefficients and $\eta$ the arguments of an objective function that computes $\boldsymbol{W}$ using the input spline coefficients, 
calls the model-fitting functions \texttt{lmer} and/or \texttt{glmer} to fit the other parameters in the survival and growth regressions, 
and returns an approximate AIC value and model degrees of freedom ($df$) for survival and growth combined. We used the $\vec{b}$ values at the smoothest 
local minimum of AIC as a function of $df$, as in \cite{teller_linking_2016}. This approach assumes that one measure of crowding affects 
survival and growth. In addition, for fitting the kernels we assumed that survival and growth depended only on intraspecific crowding, and thus only fitted the
within-species competition kernels $F_{jj}$. Based on previous work \citep{adler_coexistence_2010}, we set all $F_{mj}$ equal to $F_{jj}$, meaning that 
the within-species competition kernel for species $j$ is also used to determine the effect of all other species on species $j$. We used data from all historical plots and contemporary control-treatment plots to estimate the competition kernels \citep{adler_in_prep}. 

Once we had estimated the competitions kernels, we used them to calculate the values of $\boldsymbol{W}$ for each individual, 
and fit the full survival and growth regressions, which include the interspecific interaction coefficients, $\boldsymbol{\omega}$. 
All genets in a quadrat were included in calculating $W$, but plants located within 5 cm of quadrat edges were not used in fitting. 

We model recruitment at the quadrat level rather than at the individual genet level because the mapped data do not allow us to determine which recruits were produced by which potential parent plants. We assume that the number of individuals, $y$, of species $j$ recruiting at time $t+1$ in the location $q$ follows a negative binomial distribution:
\begin{equation}
y_{jq,t+1}= NegBin(\lambda_{jq,t+1},\theta) 	   
\label{eqn:recrDataModel}
\end{equation}
where $\lambda$ is the mean intensity and $\theta$ is the size parameter. In turn, $\lambda$ depends on the composition of the quadrat in the previous year:
\begin{equation}
\lambda_{jq,t+1} = C'_{jq,t} \exp{\left(\varphi_{jg}^R + \gamma_{j,t}^R + 
\left \langle \boldsymbol{\omega}^R , \boldsymbol{\sqrt{C'}_{q,t}} \right \rangle \right) }
\label{eqn:recrProcessModel}
\end{equation}
where the superscript $R$ refers to Recruitment, $C'_{jq,t}$ is the `effective cover' (cm$^2$) of species $j$ in quadrat $q$ at time $t$, $\varphi$ is a group dependent intercept, $\gamma$ is a random year effect, $\boldsymbol{\omega}$ is a vector of coefficients that determine the strength of intra- and interspecific density-dependence, and $\boldsymbol{C'}$ is the vector of ``effective'' cover of each species in the community. Following previous work \citep{adler_coexistence_2010}, we treated year as a random factor allowing intercepts to vary among years. 
   
Because plants outside the mapped quadrat could contribute recruits to the focal quadrat or interact with plants in the focal quadrat, we estimated effective cover as a mixture of the observed cover, $C$, in the focal quadrat, $q$, and the mean cover, $\bar{C}$, across the spatial location, $g$, in which the quadrat is located: $C'_{jq,t}=p_j C_{jq,t}+(1-p_j) \bar{C}_{jg,t}$, where $p$ is a mixing fraction between 0 and 1 that was estimated as part of fitting the model.

In the treatment model, a new term is added to the exponential term in the equation above, 
\begin{equation}
 \boldsymbol{T}\boldsymbol{\chi}_{j}^R
\end{equation}
where $\chi$ describes the effect of the treatment levels on recruitment.

Likewise in the climate model this term is added,
\begin{equation}
\boldsymbol{C}\xi_{j}^R 
\label{eqn:recC}
\end{equation}
where the $\boldsymbol{\xi}$ gives a set of coefficients for the soil year, and treatment specific soil moisture covariates in $\boldsymbol{C}$.

We fit all vital rate models using Hamiltonian-Markov Chain Monte Carlo (HMCMC) simulations in STAN 6.4 \citep{stan} and  \citep{rstan}. The priors and model code are described more completely in appendix A. Each model was run for 2,000 iterations and four independent chains with different initial values for parameters. We discarded the initial 1,000 samples. Convergence was observed graphically for all parameters, and confirmed by assessing the split $\widehat{R}$ statistic which at convergence is equal to one \citep{stan manual}. 

To assess the effects of the experimental treatments on species survival and growth we fit treatment models with and without the size by treatment interactions in the treatment effect term for (\ref{eqn:survT}) and growth models (\ref{eqn:growT}). We judged whether including interaction terms improved model fit by comparing the Watanabe-Aikake Information Criteria (WAIC) scores for each version of the model and retained the version with the lower WAIC score \citep{vetari_practical_2015}. WAIC scores approximate cross-validation predictive accuracy for a model and like traditional AIC allow for model comparison. Lower WAIC scores indicate a more parsimonious model. When a treatment model for survival or growth for a species included a size by treatment effect, we also included a size by soil moisture effect in the climate model for that species and vital rate. This allowed us to more directly compare the effects in the experimental data to the effects predicted from the climate model fit to the observational data.

\subsection*{Selecting soil moisture covariates}

After generating a time series of predicted daily soil moisture from the SOILWAT model, we averaged daily soil moisture across spring, summer and fall seasons in each year. We considered each of the three seasonal soil moisture variables at three different time periods relative to the demographic transition from year $t$ to year $t+1$.  Soil moisture in the year between $t$ and $t+1$ is indicated with a "1" subscript.  Soil moisture in the year before $t$ is indicated with a "0" subscript. And soil moisture preceding this year is indicated with a "lag" subscript. For example, for the year 2010, $spring_1$ would indicate soil moisture in the spring of 2010, $spring_0$ would indicate soil moisture during the spring of 2009 and $spring_{lag}$ would indicate soil moisture during the spring of 2008.   

In order to select among the nine potential soil moisture covariates (three seasons and three lags each) for each species and vital rate, we first fit a model with random year effects but without climate covariates to the observational data up to 2010. We then extracted the mean of the year effects estimates for each fitted year in the data. These random effects represented unexplained deviations in the average vital rate in a given year.  We then found the correlations between each of the soil moisture variable and the random year effects. For each vital rate and species, we selected the three covariates with highest correlations with these year effects. This screening technique has been used in previous demographic studies at this site \citep{dalgleish_climate_2010} and is often used in dendrochronology to screen for potential climate influence on tree-ring growth \citep{wang trees}; however, it is subject to the criticism that it is a form of data dredging \citep{dredging}.  Nevertheless, we felt that this approach was justified in this study because ultimately we did not make inference from these fitted parameters until after we tested their ability to predict the data in the experimental plots.  

Once we selected soil moisture covariates for each of the species and vital rates we fit the climate models including the climate terms for survival \ref{eqn:survC}, growth \ref{eqn:growC} and recruitment \ref{eqn:recC}.  We fit these models only to the observational data from the historical period and (1928 to 1957) and the first four years of the modern period (2007 to 2010).  Thus in generating predictions for the treatment effects observed in from 2011 to 2016 we are predicting truly out of sample data.

\subsection*{Predicting cover from individual-based models}

The vital rate regressions allow us to evaluate whether soil moisture and the experimental treatments had an effect on species performance.  But the population response ultimately depends on the integrated effects of treatment or soil moisture on all three vital rates.  To evaluate whether the climate models could predict the responses of these species in the drought and irrigation experiment at the overall population level we used an individual-based model (IBM) to compare observed and predicted changes in population size from one year to the next. 

To simulate changes in cover in each quadrat from year $t$ to year $t+1$, we initialized the IBM with the observed genet sizes and locations of the four focal species observed in year $t$ in each quadrat. For every individual genet in a quadrat, we projected its size and survival probability in the next year using the growth and survival models and the appropriate crowding and soil moisture or treatment covariates for that year and quadrat.  Likewise we projected the number of new recruits in the quadrat in the next year using the recruitment model. We calculated the expected cover in year $t+1$ as the total area of new recruits, plus the sum of the predicted area of each existing plant at time $t+1$ multiplied by each plant's expected survival probability from time $t$ to $t+1$. 

We accounted for the uncertainty in our random year effects when generating predictions, by drawing random year effects for each predicted year from a normal distribution with a mean of zero and a standard deviation drawn from the posterior estimate of the standard deviation of the random year effects \ref{eqn:}. We generated predictions using the full posterior distributions of each model parameter which allowed us to carry forward all of the uncertainty of the fitted vital rate models into our cover predictions. Because we were interested in comparing model predictions to observations, and were not interested in the effects of demographic stochasticity, we used a deterministic version of the models (e.g., recruitment is the $\lambda$ of (\ref{eqn:recrProcessModel}), rather than a random draw from a negative binomial distribution with a mean of $\lambda$).

We generated predictions for the full time series of observations, including the five experimental years, from the year effects model, the climate model and the treatment model. After generating predictions for each year, we found the mean cover across all quadrats in each treatment level and then calculated the predicted log cover change as $log(Cover_{t+1}/Cover_{t})$.

\subsection*{Quantifying predictive accuracy} 
After fitting the year effects and climate models to the observational data, we generated predictions from these models for each of the vital rates in the experimental plots. We then assessed performance of the climate and null models by calculating the mean square error (MSE) between the predicted and observed responses in the experimental data as, 

\begin{equation}
MSE = \frac{1}{n} \sum_{i=1}^{n} (y_i - E(y_i|\theta))^2, 
\label{eqn:MSE}
\end{equation}
where $y_i$ is the outcome of observation $i$ and $E(y_i|\theta)$ gives the expected outcome given the parameters in the model $\theta$. The MSE is easy to interpret, but is not always appropriate for models fit with non-normal error structures \citep{gelman_understanding_2014}. A more general statistic for assessing model predictions is the log pointwise predictive density (lppd) \citep{gelman_understanding_2014}.  The lppd for a given model is defined as, 

\begin{equation}
lppd = \sum_{i=1}^{n} log \int p(y_i| \theta)p_{post}(\theta) d\theta, 
\label{eqn:lppd}
\end{equation}
where the integral on the right side gives the probability of observing the outcome $y$ at each data point $i$ given the full posterior distribution of the parameters in the model $p_{post}(\theta)$. In practice we computed the lppd from the posterior simulations generated by STAN as, 

\begin{equation}
\widehat{lppd} = \sum_{i=1}^{n} log \left(\frac{1}{S} \sum_{s=1}^{S} p(y_i | \theta^S) \right),
\label{eqn:clppd}
\end{equation}
where the summation of $p(y_i|\theta^S)$ gives the total probability of observing the the actual response $y_i$ given the simulated posterior distribution $\theta^S$ across the full set of model simulations $S$.  The log of this sum is then averaged across the set of all observations $i$.  Higher lppd scores indicate that the model better predicts the observations.

In addition, we evaluated whether the climate model predicted treatment effects of similar direction and magnitude to those observed in the experiment.  We did this by extracting the soil moisture coefficients contained in $\xi$ for each of the vital rates and then multiplying those by the appropriate soil moisture covariates for each year and treatment level in the experiment.  We then averaged these across all five years in the experiment to find the average treatment effect predicted by the climate model.  We compared these to the posteriors of the treatment parameters, $chi$, from the treatment model.  As a measure of agreement between our predictions and observed response we calculated the correlation between the predicted and observed parameter values. 

We considered the effect of climate covariates or treatment effects to be significant when the 95\% Bayesian credible intervals on the posterior estimates did not overlap zero.  

All data and R code necessary to reproduce our analysis will be deposited in the Dryad Digital Repository once the manuscript is accepted. The current version of the computer code is available at https://github.com/pbadler/ExperimentTests/tree/master/precip and the data are available at https://bitbucket.org/ellner/driversdata. 

\section*{Results}

\subsection*{Effects on soil moisture}

Our treatments successfully changed the soil moisture in the experimental plots in the directions expected (fig. \ref{fig:spotVWC}). Spring spot measurements of soil moisture from all the plots showed that on average the drought plots were roughly 50\% drier, while irrigated plots were roughly 40\% wetter than ambient conditions (table \ref{table:spotVWC}).
  
The continuously recorded soil moisture data also showed treatment effects on soil moisture, but these effects were weaker on average than the spot measurements and depended on season and recent rainfall (table \ref{table:soil_moisture_model}; fig \ref{fig:dailyVWC}). We saw weaker effects during the spring than during the fall and summer: the drought plots were about 20-30\% drier in the fall and summer but only 7 to 14\% drier during the spring, while the irrigated plots were 30\% wetter during the fall and summer but only 20-25\% wetter during the spring.  Treatment differences were slightly larger during rainy periods, especially in the spring. We did not find evidence that the drought shelters and the irrigation treatments consistently affected air temperature at 30 cm above the plots.  

The SOILWAT soil moisture model predicted average monthly soil volumetric water content of between 10 ml/ml and 15 ml/ml each month, with the month of April being the wettest and the month of July, August and September being the driest on average. Annual variation in seasonal soil moisture for each year was positively correlated with seasonal precipitation and negatively correlated with seasonal temperature. During the course of the experiment, SOILWAT reproduced much of the daily variation observed in soil moisture recorded by our automatic data loggers, but the average soil moisture predicted by SOILWAT was about 5 ml/ml wetter than the soil moisture content recorded by the data loggers.  

After adjusting the SOILWAT seasonal soil moisture predictions by the treatment effects, we found that the soil moisture predicted in the drought plots during the course of the experiment was well below the historical seasonal averages: the summer of 2012 and 2013, the fall of 2013, and the spring and winter of 2014 fell below the 5th percentile limit for drought in the historical period \ref{fig:seasonalVWC}. Soil moisture in our irrigation plots was generally above the historical average soil moisture but conditions never exceeded the 90th percentile for soil moisture in the historical period. 

\subsection*{Effects on cover and vital rates}

The cover of \textit{H. comata} and \textit{P. spicata} fell significantly in the drought plots from 2011 to 2016 (tables \ref{table:changeHECO}, \ref{table:changePSSP}, fig \ref{fig:coverChange}). The cover of \textit{P. secunda} showed a slight decrease in the drought plots and an increase in the irrigated plots but these changes were not significant (table \ref{table:changePOSE}).  In contrast to the grasses, the cover of \textit{A. tripartita} increased slightly in all three treatments (fig \ref{fig:coverChange}). 

Our treatment models fit to the experimental and observational data indicated a variety of treatment effects on the vital rates of each species. After comparing WAIC scores for the growth models with and without the size by treatment effects, we retained size by treatment interaction effects in the growth models for \textit{A. tripartita} and \textit{P. secunda}, and the survival model for \textit{P. secunda}. For \textit{A. tripartita} we found significant size by treatment effects of drought: drought had positive effects on plants of average size and smaller \ref{fig:growthTreat}, but plants larger than the mean size by more than 1.5 standard deviations grew slightly less in the drought treatment than in the controls. \textit{A. tripartita} showed the opposite response in the irrigated plots, (although the irrigation parameters were not technically significant at the 95\% confidence level): irrigation reduced growth for small plants while irrigation increased growth of plants more than 1.5 standard deviations greater than the mean size. Drought led to a strong (but not signficant) decrease in \textit{H. comata} growth, while irrigation had no effect on growth.  Like \textit{A. tripartita}, we saw size by treatment effects on \textit{P.secunda} growth, with the negative effects of drought becoming greater for larger plants. \textit{P. secunda} showed the opposite response in the irrigation plots with larger plants showing the largest increase in growth in response to irrigation (although these effects were technically not significant). \textit{P. spicata} growth was relatively unaffected by the drought and irrigation treatments. 

Drought decreased the survival of all three grass species (fig \ref{fig:survivalTreat}). And \textit{P. secunda} showed a negative size by drought interaction effect, indicating that the survival of larger plants was more negatively affected by drought than that of the smaller plants, with the smallest plants (plants one standard deviation smaller than the mean) actually seeing a slightly positive effect of drought. \textit{A. tripartita} survival was relatively unaffected by the drought and irrigation treatments.

Recruitment in our irrigation plots was significantly less than in control plots for two grass species \textit{P. secunda} and \textit{P. spicata} (fig \ref{fig:recruitmentTreat}). However, recruitment was also lower in the drought plots than in the the control plots (although not significantly so), indicating that the decrease in the irrigated plots may have not been entirely due to the irrigation itself. The recruitment data for \textit{A. tripartita} were relatively limited, with only 32 new recruits in total observed in all 30 plots over the course of the five year experiment.

As expected from previous research most of our demographic models estimated strong negative intra-specific crowding effects and weaker negative inter-specific crowding effects on the focal species (appendix) \citep{adler_coexistence_2010,chu_direct_2016,chu_large_2015}.

\subsection*{Effects of soil moisture on vital rates}

We choose three seasonal soil moisture variables for each species' climate model based on their correlation with the year effects in the random year effects model fit to the observational data (table \ref{table:strongCor}). We included size by soil moisture variables for \textit{A. tripartita} and  \textit{P. secunda} based on the treatment response we observed in the experiment. All three time lags and all three seasons show up in the choosen variables. After fitting the vital rate models with the selected soil moisture variable we observed a trend towards positive soil moisture effects on growth of all three grasses \ref{fig:climateGrowth}. For \textit{H.comata} the soil moisture of the most recent summer had a significantly positive effect while the soil moisture of the previous summer and the fall before that were also positive but not significant. For \textit{A. tripartita} the summer and fall soil moisture of the previous year had strong negative effects on growth.  There were also strong positive size by climate interaction effects for these variables: soil moisture had a stronger negative effect on small plants and a positive effect only on the largest plants.

Soil moisture had significant effects on the survival of all four species \ref{fig:climateSurvival}. As for growth the grasses showed mainly positive effects while \textit{A. tripartita} showed a significant negative effect of soil moisture in the summer of the previous year and a strong negative effect of spring soil moisture of the previous year. \textit{H. comata} showed a significant positive effect of spring lag soil moisture and a strong positive effect of spring soil moisture in the previous year. \textit{P. secunda} showed a significant positive effect of the previous spring's soil moisture and there was a significant interaction between this effect and plant size: as plant size increased this effect became more strongly positive. Finally for \textit{P. spicata} there was a significant positive effect of lag spring soil moisture on survival.

There were only two significant effects of soil moisture on recruitment: lag fall soil moisture had a positive effect on \textit{P. secunda}, and lag summer soil moisture had a negative effect on \textit{P. spicata} recruitment. Soil moisture of the previous year's summer had also had a strong negative effect on \textit{P. spicata} recruitment.  

The intra- and interspecific crowding effects estimated in the climate model were similar to those estimated in the treatment model (appendix) \citep{adler_coexistence_2010,chu_direct_2016,chu_large_2015}.

\subsection*{Evaluating the predictions}

For most models adding climate covariates did not improve our ability to predict species responses in the experiment \ref{table:overallPreds}. However, the climate models did improve overall prediction MSE for growth of \textit{A. tripartita} and growth and survival of \textit{P. secunda} (table \ref{table:overallPreds}). In terms of lppd, the climate model outperformed the year effects model for \textit{A. tripartita} growth, \textit{H. comata} recruitment, \textit{P. secunda} growth and survival and \textit{P. spicata} recruitment.  .

When we look at the predictions for each treatment separately we see that climate covariates improved model predictions more often in the drought treatments than in the control or irrigation treatments \ref{table:treatmentPreds}. For all four species, the climate model outperformed the year effects model for predicting the response of growth to drought in terms of lppd \ref{table:treatmentPreds}. The climate model outperformed the year effects model for predicting irrigation effects on growth for all species except \textit{H. comata}. 

Overall our climate models often predicted the correct direction of the drought and irrigation treatments \ref{fig:parPredictions}. In four cases we both observed and predicted treatment effects significantly different from zero based on the 95\% Bayesian credible interval around the parameters: the drought response of \textit{H. comata} survival (fig \ref{fig:parPredHECOSurvival}), the drought response of \textit{P. secunda} growth (fig \ref{fig:parPredPOSEGrowth}), the irrigation response of \textit{P spicata} recruitment (fig \ref{fig:parPredPSSPRecruitment}) and the irrigation response of \textit{P. secunda} recruitment (fig \ref{fig:parPredPOSERecruitment}).  In only one of these cases, for \textit{P. secunda} recruitment, was the predicted effect in the opposite direction from the observed treatment effect \ref{fig:parPredictions}. The overall correlation between the predicted and observed treatment effects for all treatments, species and vital rates was r = 0.54, whereas the correlation for the drought treatment effects (r = 0.77) was better than for the irrigation effects (r = 0.46).  Also for the three models in which we included size by treatment or size by climate interactions the correlation of these size dependent effects were much stronger than the intercept parameter estimates \ref{fig:parPredictions}.  

Using the vital rate models for each species we generated one year ahead cover predictions for each quadrat in each year of the experiment.  Average cover predicted by the climate model tended to be lower than the observed cover each year for \textit{A. tripartita} and \textit{P.secunda} (fig \ref{fig:coverPred}). Comparing the overall population growth rates predicted to those observed in the experiment, we see that the MSE of the climate model was lower than the MSE of the year effects model for \textit{P. secunda} and \textit{P. spicata} (table \ref{table:corPGR}). The predictions produced by the climate model for these species were also slightly more correlated with the observations than the predictions produced by the year effects model (table \ref{table:corPGR}). Considering each treatment and species separately we see that the predicted population growth rates for \textit{A. tripartita}, \textit{P. secunda} and \textit{P. spicata} were all consistently lower than the observed population growth rates (figs \ref{fig:pgrARTR}, \ref{fig:pgrPOSE}, \ref{fig:pgrPSSP}). The climate model showed lower MSE for \textit{A. tripartita}, \textit{P. secunda} and \textit{P. spicata} in the irrigation treatment, \textit{P. spicata} in the control treatment and \textit{H. comata} in the drought treatment (figs \ref{fig:pgrHECO}).  However, the correlations between the predicted and observed log changes in cover did not always show the same pattern as MSE: the climate model made more strongly correlated predictions with the observations than the year effects model only for \textit{P. spicata} and \textit{P. secunda} in the control treatment and \textit{P. secunda} and \textit{H. comata} in the drought treatment. 
  

\section*{Discussion}

Our experiment showed that observational data on the response of plant populations to interannual climate variation can indeed help us predict the direction of species response to experimental climate manipulations \ref{fig:parPredictions}. This was true even though adding climate parameters to the demographic models only improved vital rate predictions for half of the models \ref{table:overallPreds}. This should give us some hope that even when climate effects in demographic models to observational data are weak, they may contain useful qualitative information on the direction of climate effects in the future. 

* discuss biggest successes--astreces in fig \ref{fig:parPredictions}
* discuss biggest mistake, P. secunda recruitment went in wrong direction. 
 
Scaling up to the population level, the climate models only produced better one step a head predictions of the overall response of the species to the experiment for two species: \textit{P. spicata} and \textit{P. secunda} (table \ref{table:corPGR}). However adding climate effects produced better population-level predictions in the drought treatments for \textit{P. secunda} and \textit{H. comata}.  Both the survival and the growth of these species were positively affected by increased soil moisture (fig \ref{fig:climateGrowth}, \ref{fig:climateSurvival}), and so it makes intuitive sense that we would observe declines in these species cover in the drought treatment (fig \ref{fig:coverChange}). 

*Among our species, we had the most success predicting the response of the three grass species and less success predicting the response of the shrub species \textit{A. tripartita}.    Why might ARTR be different? 

*We also had more success predicting species response to the drought treatment than to the irrigation treatment. We also tended to see species respond more strongly to the drought treatment than they did to the irrigation treatment.  This tells us that in some cases, demographic models based on observational data may be of more use for prediction in extreme conditions than when conditions are close to average.  Our drought treatment likely created extremely low soil moisture compared to the historical average \ref{fig:seasonalVWC}. We hypothesize that this made water the most limiting resource for plants in these plots. In contrast, in the control and irrigated plots soil moisture may not have always been limiting during the course of the experiment.  It may make sense then that our climate models did not make more accurate predictions in these conditions than a random year effects model.

*Mixed success at predicting species responses to high and low moisture availability has interesting implications.  On the one hand it is re-assuring that observational data is sometimes useful. On the other hand, for predictions to be truly useful we would also like more information to help us sort out why our predictions for some species were good and for others no better than a null model.  In other words, we have little ability to predict when our predictions are likely to be accurate.  Among plants detailed physiological ecology may give us a guide to which factors and climate conditions are likely to affect which species.  In our system, for instance, it may make sense that the grasses showed a stronger response to the drought treatment than the a woody shrub.  Although these grasses are adapted to the arid conditions that characterize the sagebrush steppe, they thrive during the brief window in spring and early summer when the soil moisture and temperatures are warmer.  \textit{A. tripartita} on the other hand grows throughout the summer and generally has deeper roots than the grasses. These traits may help it tolerate the water stress induced by drought. 

*Will our predictions be useful for the longer term? Because these species compete, one could argue that our predictions for any one species in this community will only be as good as the predictions we make for their competitors. For instance, while we observed little effect of drought on \textit{A. tripartita} in our experiment, it is possible that it will eventually respond positively to the drought treatment as cover of the grass species it comptetes with declines (fig \ref{fig:coverChange}). We know that grass species compete strongly with \textit{A. tripartita}.  However, we also know from previous work in this community that each species is more limited by intra-specific competition than by inter-specific crowding.  This fact ensures that the direct effects of climate change will generally be greater than the competition mediated indirect effects of climate change \cite{chu_direct_2016, kleinhesselink_indirect_2015}.  

*Our results give us more confidence that historical observational data can in theory be used to detect and predict the demographic effects of climate change. This should encourage more researchers to try and use observational data to predict the future in both experimental and natural settings. Nevertheless, our success at predicting the short-term response of two species to a small-scale climate manipulation is not likely to be very reassuring to applied ecologists and resource managers wishing to make accurate medium to long-term quantitative predictions about the effects of climate change on the species they manage.  Clearly more work is needed to distinguish which predictions are worth having confidence in and which predictions we should have less confidence in. Towards that goal, perhaps the best way to build confidence in ecological predictions is conduct more tests like this one.   
  
\section*{Acknowledgements}

Funding was provided by an NSF GRFP to AK, NSF grants DEB-1353078, and DEB-1054040 to PBA and by the Utah Agricultural Experiment Station (get journal paper number). The USDA-ARS Sheep Experiment Station generously provided access to historical data and the field experiment site. Joe and Kevin, Lau, John Bradford, Caitlin (soilwat), 

\newpage
\bibliographystyle{Ecology}
\bibliography{precip_experiment}


\end{doublespacing} 

\clearpage
\newpage

\section*{Tables}

\begin{table}[h]
	\caption{Treatment effects on spring soil moisture}
	\centering
	%\begin{center}
	\begin{tabular}{l c c }
		\hline
		& Model 1 \\
		\hline
		(Intercept)                  & $8.81^{*}$    	& $[5.78;\ 11.83]$  \\
		TreatmentDrought             & $-3.97^{*}$      & $[-4.84;\ -3.09]$ \\
		TreatmentIrrigation          & $3.26^{*}$       & $[2.39;\ 4.14]$   \\
		\hline
		AIC                          & 3191.87           \\
		BIC                          & 3222.92           \\
		Log Likelihood               & -1588.93          \\
		Num. obs.                    & 624               \\
		Num. groups: plot            & 24                \\
		Num. groups: PrecipGroup     & 8                 \\
		Num. groups: date            & 5                 \\
		Var: plot (Intercept)        & 0.45              \\
		Var: PrecipGroup (Intercept) & 0.23              \\
		Var: date (Intercept)        & 11.24             \\
		Var: Residual                & 8.90              \\
		\hline
		\multicolumn{2}{l}{\scriptsize{$^*$ 0 outside the confidence interval}}
	\end{tabular}
	\label{table:spotVWC}
	%\end{center}
\end{table}

\begin{table}
	\caption{Model of treatment effects on soil moisture}
	\begin{center}
		\begin{tabular}{l c c }
			\hline
			& Model 1 \\
			\hline
			(Intercept)                       & $-0.57^{*}$       & $[-0.89;\ -0.26]$ \\
			TreatmentIrrigation               & $1.23^{*}$        & $[1.18;\ 1.29]$   \\
			rainfallrainy                     & $-0.05$           & $[-0.12;\ 0.01]$  \\
			seasonspring                      & $0.27^{*}$        & $[0.23;\ 0.32]$   \\
			seasonsummer                      & $0.15^{*}$        & $[0.10;\ 0.19]$   \\
			seasonwinter                      & $0.25^{*}$        & $[0.21;\ 0.29]$   \\
			TreatmentIrrigation:rainfallrainy & $0.18^{*}$        & $[0.13;\ 0.24]$   \\
			TreatmentIrrigation:seasonspring  & $-0.23^{*}$       & $[-0.29;\ -0.16]$ \\
			TreatmentIrrigation:seasonsummer  & $-0.26^{*}$       & $[-0.33;\ -0.20]$ \\
			TreatmentIrrigation:seasonwinter  & $-0.33^{*}$       & $[-0.40;\ -0.27]$ \\
			rainfallrainy:seasonspring        & $-0.23^{*}$       & $[-0.31;\ -0.16]$ \\
			rainfallrainy:seasonsummer        & $-0.07$           & $[-0.14;\ 0.01]$  \\
			rainfallrainy:seasonwinter        & $-0.07$           & $[-0.20;\ 0.07]$  \\
			\hline
			AIC                               & 14581.58          \\
			BIC                               & 14695.49          \\
			Log Likelihood                    & -7274.79          \\
			Num. obs.                         & 9133              \\
			Num. groups: date                 & 1596              \\
			Num. groups: PrecipGroup          & 8                 \\
			Var: date (Intercept)      		  & 0.00              \\
			Var: PrecipGroup (Intercept)      & 0.19              \\
			Var: Residual                     & 2.50              \\
			\hline
			\multicolumn{2}{l}{\scriptsize{$^*$ 0 outside the confidence interval}}
		\end{tabular}
		\label{table:soil_moisture_model}
	\end{center}
\end{table}

% latex table generated in R 3.3.1 by xtable 1.8-2 package
% Mon Dec 19 18:10:15 2016
\begin{table}[ht]
	\centering
	\begin{tabular}{rrrrr}
		\hline
		& Estimate & Std. Error & t value & Pr($>$$|$t$|$) \\ 
		\hline
		(Intercept) & -0.2835 & 0.4627 & -0.61 & 0.5487 \\ 
		TreatmentDrought & 0.9378 & 0.7743 & 1.21 & 0.2434 \\ 
		TreatmentIrrigation & 0.8882 & 0.7743 & 1.15 & 0.2682 \\ 
		\hline
	\end{tabular}
	\caption{Treatment effects on log cover change for \textit{A. tripartita} from 2011 to 2016. Intercept gives control effects.} 
	\label{table:changeARTR}
\end{table}

% latex table generated in R 3.3.1 by xtable 1.8-2 package
% Mon Dec 19 18:10:16 2016
\begin{table}[ht]
	\centering
	\begin{tabular}{rrrrr}
		\hline
		& Estimate & Std. Error & t value & Pr($>$$|$t$|$) \\ 
		\hline
		(Intercept) & 0.3982 & 0.2587 & 1.54 & 0.1548 \\ 
		TreatmentDrought & -2.9951 & 0.5784 & -5.18 & 0.0004 \\ 
		TreatmentIrrigation & -0.1219 & 0.4953 & -0.25 & 0.8105 \\ 
		\hline
	\end{tabular}
	\caption{Treatment effects on log cover change for \textit{H. comata} from 2011 to 2016. Intercept gives control effects.} 
	\label{table:changeHECO}
\end{table}

% latex table generated in R 3.3.1 by xtable 1.8-2 package
% Mon Dec 19 18:10:16 2016
\begin{table}[ht]
	\centering
	\begin{tabular}{rrrrr}
		\hline
		& Estimate & Std. Error & t value & Pr($>$$|$t$|$) \\ 
		\hline
		(Intercept) & -0.7247 & 0.4613 & -1.57 & 0.1298 \\ 
		TreatmentDrought & 0.0273 & 0.8208 & 0.03 & 0.9737 \\ 
		TreatmentIrrigation & 1.1459 & 0.7797 & 1.47 & 0.1552 \\ 
		\hline
	\end{tabular}
	\caption{Treatment effects on log cover change for \textit{P. secunda} from 2011 to 2016. Intercept gives control effects.} 
	\label{table:changePOSE}
\end{table}

% latex table generated in R 3.3.1 by xtable 1.8-2 package
% Mon Dec 19 18:10:16 2016
\begin{table}[ht]
	\centering
	\begin{tabular}{rrrrr}
		\hline
		& Estimate & Std. Error & t value & Pr($>$$|$t$|$) \\ 
		\hline
		(Intercept) & 0.0188 & 0.2124 & 0.09 & 0.9303 \\ 
		TreatmentDrought & -0.8851 & 0.3780 & -2.34 & 0.0287 \\ 
		TreatmentIrrigation & 0.1453 & 0.3780 & 0.38 & 0.7044 \\ 
		\hline
	\end{tabular}
	\caption{Treatment effects on log cover change for \textit{P. spicata} from 2011 to 2016. Intercept gives control effects.} 
	\label{table:changePSSP}
\end{table}



% latex table generated in R 3.3.1 by xtable 1.8-2 package
% Tue Dec 20 12:39:32 2016
\begin{table}[ht]
	\centering
	\begin{tabular}{lllrrrr}
		\hline
		vital\_rate & species & climate variable & Int. cor. & p val. & Size cor. & Size p. val. \\ 
		\hline
		growth & ARTR & su.0 & -0.49 & 0.02 & 0.26 & 0.26 \\ 
		growth & ARTR & f.0 & -0.28 & 0.23 & 0.40 & 0.08 \\ 
		growth & ARTR & sp.1 & 0.17 & 0.45 & -0.33 & 0.14 \\ 
		growth & HECO & su.1 & 0.69 & 0.00 &  &  \\ 
		growth & HECO & su.0 & 0.50 & 0.02 &  &  \\ 
		growth & HECO & f.lag & 0.37 & 0.10 &  &  \\ 
		growth & POSE & f.lag & 0.31 & 0.17 & -0.11 & 0.64 \\ 
		growth & POSE & su.lag & 0.29 & 0.20 & -0.20 & 0.38 \\ 
		growth & POSE & sp.1 & 0.26 & 0.25 & -0.20 & 0.38 \\ 
		growth & PSSP & f.lag & 0.34 & 0.13 &  &  \\ 
		growth & PSSP & su.lag & 0.25 & 0.27 &  &  \\ 
		growth & PSSP & f.0 & -0.22 & 0.34 &  &  \\ 
		recruitment & ARTR & su.lag & -0.32 & 0.16 &  &  \\ 
		recruitment & ARTR & su.0 & -0.26 & 0.25 &  &  \\ 
		recruitment & ARTR & sp.1 & 0.22 & 0.34 &  &  \\ 
		recruitment & HECO & su.lag & -0.31 & 0.18 &  &  \\ 
		recruitment & HECO & su.0 & -0.30 & 0.18 &  &  \\ 
		recruitment & HECO & f.lag & 0.19 & 0.40 &  &  \\ 
		recruitment & POSE & sp.1 & 0.49 & 0.02 &  &  \\ 
		recruitment & POSE & f.lag & 0.34 & 0.13 &  &  \\ 
		recruitment & POSE & f.1 & 0.32 & 0.16 &  &  \\ 
		recruitment & PSSP & su.lag & -0.52 & 0.02 &  &  \\ 
		recruitment & PSSP & su.0 & -0.48 & 0.03 &  &  \\ 
		recruitment & PSSP & sp.0 & 0.30 & 0.19 &  &  \\ 
		survival & ARTR & su.0 & -0.60 & 0.00 &  &  \\ 
		survival & ARTR & sp.0 & -0.41 & 0.06 &  &  \\ 
		survival & ARTR & su.1 & -0.40 & 0.07 &  &  \\ 
		survival & HECO & sp.0 & 0.44 & 0.04 &  &  \\ 
		survival & HECO & sp.lag & 0.43 & 0.05 &  &  \\ 
		survival & HECO & f.1 & 0.33 & 0.14 &  &  \\ 
		survival & POSE & sp.0 & 0.44 & 0.04 & 0.22 & 0.34 \\ 
		survival & POSE & sp.1 & 0.27 & 0.23 & -0.46 & 0.04 \\ 
		survival & POSE & f.lag & -0.00 & 0.99 & 0.30 & 0.19 \\ 
		survival & PSSP & sp.0 & 0.36 & 0.11 &  &  \\ 
		survival & PSSP & sp.lag & 0.34 & 0.13 &  &  \\ 
		survival & PSSP & su.1 & 0.26 & 0.26 &  &  \\ 
		\hline
	\end{tabular}
	\caption{Selected climate variables for each vital rate model for each species. Correlations and p-values between the choosen variables and the intercept of year effects model are shown. For ARTR growth and POSE growth and survival, the correlations between the year effects on size and the soil moisture variables are also given. "f" = fall, "su" = summer, "sp" = spring. ARTR = \textit{A. tripartita}, HECO = \textit{H. comata}, POSE = \textit{P. secunda}, PSSP = \textit{P. spicata}.} 
	\label{table:strongCor}
\end{table}



% latex table generated in R 3.3.1 by xtable 1.8-2 package
% Tue Dec 20 12:38:30 2016
\begin{table}[ht]
	\centering
	\begin{tabular}{lllrrrl}
		\hline
		species & vital\_rate & score & climate model & year effects model & diff & improved \\ 
		\hline
		ARTR & growth & lppd & -186.08 & -189.17 & 3.09 & *** \\ 
		ARTR & growth & MSE & 0.55 & 0.55 & -0.00 & *** \\ 
		ARTR & recruitment & lppd & -79.05 & -77.81 & -1.24 &  \\ 
		ARTR & recruitment & MSE & 169.86 & 9.47 & 160.39 &  \\ 
		ARTR & survival & lppd & -37.55 & -36.42 & -1.13 &  \\ 
		ARTR & survival & MSE & 0.06 & 0.06 & 0.00 &  \\ 
		HECO & growth & lppd & -475.54 & -454.36 & -21.18 &  \\ 
		HECO & growth & MSE & 1.26 & 1.18 & 0.09 &  \\ 
		HECO & recruitment & lppd & -149.43 & -151.60 & 2.17 & *** \\ 
		HECO & recruitment & MSE & 288.16 & 235.57 & 52.59 &  \\ 
		HECO & survival & lppd & -158.99 & -147.06 & -11.93 &  \\ 
		HECO & survival & MSE & 0.12 & 0.11 & 0.01 &  \\ 
		POSE & growth & lppd & -1823.71 & -1831.56 & 7.85 & *** \\ 
		POSE & growth & MSE & 1.72 & 1.73 & -0.01 & *** \\ 
		POSE & recruitment & lppd & -260.05 & -257.46 & -2.59 &  \\ 
		POSE & recruitment & MSE & 45.40 & 37.32 & 8.08 &  \\ 
		POSE & survival & lppd & -698.06 & -718.60 & 20.54 & *** \\ 
		POSE & survival & MSE & 0.14 & 0.14 & -0.00 & *** \\ 
		PSSP & growth & lppd & -1232.93 & -1237.92 & 4.99 & *** \\ 
		PSSP & growth & MSE & 1.51 & 1.51 & 0.00 &  \\ 
		PSSP & recruitment & lppd & -271.34 & -273.94 & 2.60 & *** \\ 
		PSSP & recruitment & MSE & 79.09 & 42.68 & 36.41 &  \\ 
		PSSP & survival & lppd & -332.47 & -307.26 & -25.21 &  \\ 
		PSSP & survival & MSE & 0.11 & 0.10 & 0.01 &  \\ 
		\hline
	\end{tabular}
	\caption{Comparison of model predictions from climate model and year effects model for each species and vital rate.  Two prediction scores are reported, MSE and lppd. Lower MSE indicates improved predictions whereas higher lppd indicates improved predictions.  Instances where the climate model outperformed the random year effects model are marked with "***" in the last column. ARTR = \textit{A. tripartita}, HECO = \textit{H. comata}, POSE = \textit{P. secunda}, PSSP = \textit{P. spicata}.} 
	\label{table:overallPreds}
\end{table}


% latex table generated in R 3.3.1 by xtable 1.8-2 package
% Tue Dec 20 19:24:15 2016
\begin{table}[ht]
	\centering
	\begin{tabular}{rllrrrl}
		\hline
		& species & stat & year effects model & climate model & diff & improved \\ 
		\hline
		1 & ARTR & cor & 0.48 & 0.19 & -0.29 &  \\ 
		2 & ARTR & MSE & 0.30 & 0.30 & 0.00 &  \\ 
		3 & HECO & cor & 0.29 & 0.22 & -0.07 &  \\ 
		4 & HECO & MSE & 0.49 & 0.57 & 0.07 &  \\ 
		5 & POSE & cor & 0.45 & 0.53 & 0.07 & *** \\ 
		6 & POSE & MSE & 0.42 & 0.41 & -0.01 & *** \\ 
		7 & PSSP & cor & 0.36 & 0.38 & 0.03 & *** \\ 
		8 & PSSP & MSE & 0.39 & 0.39 & -0.01 & *** \\ 
		\hline
	\end{tabular}
	\caption{MSE of predicted log cover changes and correlations between log cover changes predicted and observed. Predictions for the cover changes in the experimental plots were generated either from the year effects or the climate models. Instances where the climate model made better predictions than the year effects model are indicated with the "***". ARTR = \textit{A. tripartita}, HECO = \textit{H. comata}, POSE = \textit{P. secunda}, PSSP = \textit{P. spicata}.} 
	\label{table:corPGR}
\end{table}


\clearpage
\newpage


\section*{Figures}


\begin{figure}[!htbp]
	\centering
	\includegraphics[width=1\textwidth]{VWC_spot_measurements}
	\caption{Soil moisture in the upper 5 cm of drought and irrigated plots compared to ambient controls. Soil moisture was measured at six locations around each plot at five different dates during the spring. Control plots were nearby areas of experiencing ambient soil moisture. Box plots show the median soil moisture and the interquartile range.  Dots show individual soil moisture measurements. Readings of volumetric soil moisture less than zero were occasionally obtained in very dry soil.}
	\label{fig:spotVWC}
\end{figure}


\begin{figure}[!htbp]
	\centering
	\includegraphics[width=1\textwidth]{avg_daily_soil_moisture}
	\caption{Average soil moisture in the control, drought, and irrigation treatments during each year of the experiment.  Soil moisture was monitored in four drought plots, four irrigated plots and four ambient control plots. Two sensors were installed at 5 cm depth at each plot and two at 25 cm and data was logged every 2 hours.}
	\label{fig:dailyVWC}
\end{figure}


\begin{figure}[!htbp]
	\centering
	\includegraphics[width=1\textwidth]{modern_soil_moisture_comparison}
	\caption{Average seasonal soil moisture in the control, drought, and irrigation treatments during each year of the experiment. The dashed gray lines give the 5th percentile and 95th percentile limits for seasonal soil moisture in the historical record (1929 to 2010). }
	\label{fig:seasonalVWC}
\end{figure}


\begin{figure}[!htbp]
	\centering
	\includegraphics[width=1\textwidth]{start_to_finish_cover_change}
	\caption{Log change in cover in each of the experimental plots from the pre-treatment monitoring in 2011 to the last year of the experiment in 2016. Box plots show the median cover change and the interquartile range. ARTR = \textit{A. tripartita}, HECO = \textit{H. comata}, POSE = \textit{P. secunda}, PSSP = \textit{P. spicata}.}
	\label{fig:coverChange}
\end{figure}

\begin{figure}[!htbp]
	\centering
	\includegraphics[width=1\textwidth]{treatment_effect_growth}
	\caption{Parameter estimates for the effects of treatment on growth of all four species. We assessed a parameter as significant when the 95\% Bayesian credible intervals did not overlap zero. Size by treatment interactions were only fit for ARTR, and POSE. Plant size was centered on mean size and scaled by its standard deviation. ARTR = \textit{A. tripartita}, HECO = \textit{H. comata}, POSE = \textit{P. secunda}, PSSP = \textit{P. spicata}. }
	\label{fig:growthTreat}
\end{figure}

\begin{figure}[!htbp]
	\centering
	\includegraphics[width=1\textwidth]{treatment_effect_survival}
	\caption{Parameter estimates for the effects of treatment on survival of all four species. We assessed a parameter as significant when the 95\% Bayesian credible intervals did not overlap zero. Size by treatment interactions were only fit for POSE. Plant size was centered on mean size and scaled by its standard deviation.  ARTR = \textit{A. tripartita}, HECO = \textit{H. comata}, POSE = \textit{P. secunda}, PSSP = \textit{P. spicata}. }
	\label{fig:survivalTreat}
\end{figure}

\begin{figure}[!htbp]
	\centering
	\includegraphics[width=1\textwidth]{treatment_effect_recruitment}
	\caption{Parameter estimates for the effects of treatment on recruitment of all four species. We assessed a parameter as significant when the 95\% Bayesian credible intervals did not overlap zero.  ARTR = \textit{A. tripartita}, HECO = \textit{H. comata}, POSE = \textit{P. secunda}, PSSP = \textit{P. spicata}. }
	\label{fig:recruitmentTreat}
\end{figure}

\begin{figure}[!htbp]
	\centering
	\includegraphics[width=1\textwidth]{climate_effect_growth}
	\caption{Parameter estimates for the selected seasonal soil moisture covariates on the growth of all four species. Parameters are ordered chronologically from most recent to the current growing season on the right to most distant on the left. Dark blue parameters show size x climate interaction effects. We assessed a parameter as significant when the 95\% Bayesian credible intervals did not overlap zero.  ARTR = \textit{A. tripartita}, HECO = \textit{H. comata}, POSE = \textit{P. secunda}, PSSP = \textit{P. spicata}. }
	\label{fig:climateGrowth}
\end{figure}

\begin{figure}[!htbp]
	\centering
	\includegraphics[width=1\textwidth]{climate_effect_survival}
	\caption{Parameter estimates for the selected seasonal soil moisture covariates on the survival of all four species. Parameters are ordered chronologically from most recent to the current growing season on the right to most distant on the left. Dark blue parameters show size x climate interaction effects. We assessed a parameter as significant when the 95\% Bayesian credible intervals did not overlap zero.  ARTR = \textit{A. tripartita}, HECO = \textit{H. comata}, POSE = \textit{P. secunda}, PSSP = \textit{P. spicata}. }
	\label{fig:climateSurvival}
\end{figure}


\begin{figure}[!htbp]
	\centering
	\includegraphics[width=1\textwidth]{climate_effect_recruitment}
	\caption{Parameter estimates for the selected seasonal soil moisture covariates on the survival of all four species. Parameters are ordered chronologically from most recent to the current growing season on the right to most distant on the left. Dark blue parameters show size x climate interaction effects. We assessed a parameter as significant when the 95\% Bayesian credible intervals did not overlap zero.  ARTR = \textit{A. tripartita}, HECO = \textit{H. comata}, POSE = \textit{P. secunda}, PSSP = \textit{P. spicata}. }
	\label{fig:climateRecruitment}
\end{figure}

\begin{figure}[!htbp]
	\centering
	\includegraphics[width=1\textwidth]{parameter_predictions}
	\caption{The treatment effects predicted by the climate model compared to the treatment effects observed for the intercept parameters (left side) and size by climate/treatment effects (right side).  The observed treatment effects come from the treatment model fitted to all the data including the five years of the experiment.  The predicted parameters come from the climate model fitted to all years of observational data but do not include the five years of the experiment (2011 to 2016).  Treatment parameters that were both observed and predicted to be significantly different from zero are shown with the "*" symbol. The correlation between predicted and observed parameters is given on each panel. \textit{P. secunda} recruitment was predicted to be positively affected by the irrigation treatment but was in fact negatively affected. The other significant effects were in the correct direction. We assessed a parameter as significant when the 95\% Bayesian credible intervals did not overlap zero.}
	\label{fig:parPredictions}
\end{figure}

\begin{figure}[!htbp]
	\centering
	\includegraphics[width=1\textwidth]{predicted_and_observed_cover}
	\caption{Observed average cover per quadrat in the experimental and control plots and one year ahead cover predictions from the climate model. Cover predictions for each year are generated from the IBM based on the observed distribution of plants in each quadrat in the current year. Quadrat cover was not predicted for the first year of the experiment in 2011.  Note the different cover scales for ARTR and the three grass species. ARTR = \textit{A. tripartita}, HECO = \textit{H. comata}, POSE = \textit{P. secunda}, PSSP = \textit{P. spicata}.}
	\label{fig:coverPred}
\end{figure}


\begin{figure}[!htbp]
	\centering
	\includegraphics[width=1\textwidth]{ARTR_predicted_pgr_comparison}
	\caption{Observed and predicted one year ahead log change in \textit{A. tripartita} cover in the experiment. Changes in cover predicted by the climate model are shown on the left and those predicted by the year effects model are shown on the right. Correlations coefficients between predictions and observations and MSE are shown for each treatment and model.}
	\label{fig:pgrART}
\end{figure}

\begin{figure}[!htbp]
	\centering
	\includegraphics[width=1\textwidth]{HECO_predicted_pgr_comparison}
	\caption{Observed and predicted one year ahead log change in \textit{H. comata} cover in the experiment. Changes in cover predicted by the climate model are shown on the left and those predicted by the year effects model are shown on the right. Correlations coefficients between predictions and observations and MSE are shown for each treatment and model. }
	\label{fig:pgrHECO}
\end{figure}

\begin{figure}[!htbp]
	\centering
	\includegraphics[width=1\textwidth]{POSE_predicted_pgr_comparison}
	\caption{Observed and predicted one year ahead log change in \textit{P. secunda} cover in the experiment. Changes in cover predicted by the climate model are shown on the left and those predicted by the year effects model are shown on the right. Correlations coefficients between predictions and observations and MSE are shown for each treatment and model.}
	\label{fig:pgrPOSE}
\end{figure}

\begin{figure}[!htbp]
	\centering
	\includegraphics[width=1\textwidth]{PSSP_predicted_pgr_comparison}
	\caption{Observed and predicted one year ahead log change in \textit{P. spicata} cover in the experiment. Changes in cover predicted by the climate model are shown on the left and those predicted by the year effects model are shown on the right. Correlations coefficients between predictions and observations and MSE are shown for each treatment and model.}
	\label{fig:pgrPSSP}
\end{figure}




\clearpage 
\newpage 


%~~~~~~~~~~~~~~~~~~~~~~~~~~~~~~~~~~~~~~~~~~~~~~~~~~~~~~~~~~~~~~~~~~~~~~~~~~~~~
% APPENDICES !
%~~~~~~~~~~~~~~~~~~~~~~~~~~~~~~~~~~~~~~~~~~~~~~~~~~~~~~~~~~~~~~~~~~~~~~~~~~~~~

\clearpage 
\newpage 

\setcounter{page}{1}
\setcounter{equation}{0}
\setcounter{figure}{0}
\setcounter{section}{0}
\setcounter{table}{0}
\renewcommand{\theequation}{SI.\arabic{equation}}
\renewcommand{\thetable}{SI-\arabic{table}}
\renewcommand{\thefigure}{SI-\arabic{figure}}
\renewcommand{\thesection}{Section SI.\arabic{section}}

\centerline{\Large \textbf{Supporting Information }}
\centerline{Adler et al., ``Weak interspecific interactions''} 

\vspace{0.4in} 

\section{Supplementary Methods} \label{suppMethods}

\subsection{Interspecific covariance in local crowding} 
We explored interspecific covariance in local crowding experienced by individual plants, by regressing the $W$ values exerted by one neighbor species, the response variable, against the $W$ values of all other species, the independent variables. Because some $W=0$, we conducted two separate regressions. First, using all $W$'s, we fitted a generalized linear model with a logit link function to evaluate whether the probability that the focal species' $W=0$ is influenced by the value of other species' $W$'s. In this model, the dependent variable is a Bernoulli variate coding for the zero or non-zero value of the focal species' crowding, and the independent variables are the $W$'s for all other species. Second, for the set of records in which the focal species has $W>0$, we performed a linear regression, where the focal species' $W$ is the dependent variable, and the other species' $W$'s are the independent variables. We repeated these regressions for each focal species. Due to large samples size, interspecific $W$ values were often statistically significant predictors of intraspecific. However, they explained very little variance. The maximum reduction in deviance for the generalized linear regressions and $R^2$ for the linear regressions were both less than 8\%. The \texttt{R} code for this analysis is included as ..\texttt{\textbackslash Wdistrib\textbackslash exploreSurvivalWs.r}.

\subsection{Mean field approximation of local crowding for the IPM} 
\citet{adler_coexistence_2010} developed a mean field approximation for local crowding when the
competition kernels are all Gaussian functions, $F_{jm}(d) = e^{-\alpha_{jm} d^2}$. The approximation is explained in 
the online SI to \citet{adler_coexistence_2010} and in section 5.3 of \citet{Ellner2016}. 
Here we explain how that approximation was modified for the IPMs in this paper, which
used fitted nonparametric competition kernels. 

For $j \ne m$ (between-species competition), overlap between individuals is allowed. The mean field approximation is 
that from the perspective of any focal plant in species $j$, individuals of species $m$ are distributed at random in space, 
independent of each other and of their size.

Consider the region between the circles of radius $x$ and $x+dx$ centered on a focal genet of species $j$. The area of this annulus
is $2 \pi x \; dx$  to leading order for $dx \approx 0$. A species $m$ genet 
in the annulus puts competitive pressure $F_{jm}(x)$ times its area on
the focal genet. The expected total competitive pressure from all such genets 
is therefore is $F_{jm}(x) 2 \pi x \; dx$ times the expected fractional cover of species $m$ in the annulus 
(fractional cover is the total area of species $m$ genets, as a fraction of the total area). The excepted fractional cover $C_m$ of species $m$
in the annulus equals its fractional cover in the habitat as a whole, because of the assumption of random distribution
spatial distributions. We therefore have $C_m  = \int e^u n_m(u,t) du/A$ where $A$ is the total area of the habitat. 
The total expected competitive pressure on a species-$j$ genet due to species $m$ is then 
\begin{equation}
W_{jm} = \int_0^\infty{C_m F_{jm}(x) 2 \pi x \; dx}  = C_m \left [2 \pi \int_0^{\infty} x F(x) \, dx \right ].
\label{eqn:wbarm}
\end{equation} 
The quantity in square brackets is a constant (that is, it only depends on what the kernel function
is) so it can be computed once and for all for each kernel used in the IPM. The integral is finite because
all fitted kernels fall to zero at a finite distance from the focal plant. 

Our kernel fitting method only uses competition kernel values at the ``mid-ring'' distances
halfway between the inner and outer radii of a series of annuli around each focal
plant, scaled so that the value at the innermost mid-ring distance equals 1. 
In the IPM we defined the kernel at other distances by linear interpolation between values at 
mid-ring distances, except that for the innermost ring a kernel value of 1 was specified at the
outer radius of the ring and at distance $x=0$. 

Now consider within-species competition. We assume that conspecifics cannot overlap. Genet shapes are irregular, but we 
nonetheless implement the no-overlap rule by assuming that a genet of log area $u_i$ is a 
circle of radius $r_i$ where $\pi r_i^2 = e^{u_i}$. The no-overlap rule is then that the centroids of two conspecific individuals 
must be separated by at least the sum of their radii. 

For any one focal genet, the no-overlap restriction on its neighbors' locations affects 
only a negligibly small part of the habitat. The expected cover of individuals in the places
where they can occur (relative to one focal plant) is thus assumed to equal their expected locations
in the habitat as a whole. 
 
Let $C_m(u)$ be the total cover of species $m$ genets of radius $r$ or smaller, 
\begin{equation}
C_m(r) = \int_L^{\log(\pi r^2)}{\! \! \! e^z n_m(z,t) \, dz} .
\label{eqn:cm}
\end{equation}
A focal genet of radius $r$ cannot have any conspecific neighbors centered 
at distances less than $r$. It can have a neighbor centered at distance $x>r$ if that neighbor's
radius is no more than $x-r$. Adding up the expected cover of all such possible neighbors
for a focal genet of radius $r$,    
\begin{equation}
W_{mm}(r) = 2 \pi \int_r^{\infty}F_{mm}(x) x C_m(x-r) \, dx
\label{eqn:wbarmr} 
\end{equation}
This integral is again finite and computable because the kernels $F$ fall to 0 at finite $x$. 

\clearpage 
\newpage  
\section{Additional Tables} 
% latex table generated in R 3.3.1 by xtable 1.8-2 package
% Tue Dec 20 13:32:53 2016
\begin{longtable}{rllllrrrl}
	\hline
	& species & vital\_rate & Treatment & score & climate model & year effects model & diff & improved \\ 
	\hline
	\endhead
	\hline
	\multicolumn{9}{l}{\footnotesize Continued on next page}
	\endfoot
	\endlastfoot
	1 & ARTR & growth & Control & lppd & -107.53 & -108.40 & 0.87 & *** \\ 
	2 & ARTR & growth & Control & MSE & 0.57 & 0.56 & 0.01 &  \\ 
	3 & ARTR & growth & Drought & lppd & -39.71 & -40.01 & 0.30 & *** \\ 
	4 & ARTR & growth & Drought & MSE & 0.53 & 0.49 & 0.03 &  \\ 
	5 & ARTR & growth & Irrigation & lppd & -38.85 & -40.75 & 1.91 & *** \\ 
	6 & ARTR & growth & Irrigation & MSE & 0.50 & 0.57 & -0.07 & *** \\ 
	7 & ARTR & recruitment & Control & lppd & -32.61 & -31.05 & -1.56 &  \\ 
	8 & ARTR & recruitment & Control & MSE & 61.88 & 10.59 & 51.29 &  \\ 
	9 & ARTR & recruitment & Drought & lppd & -28.93 & -26.52 & -2.41 &  \\ 
	10 & ARTR & recruitment & Drought & MSE & 523.95 & 11.62 & 512.33 &  \\ 
	11 & ARTR & recruitment & Irrigation & lppd & -17.51 & -20.24 & 2.73 & *** \\ 
	12 & ARTR & recruitment & Irrigation & MSE & 4.72 & 5.34 & -0.62 & *** \\ 
	13 & ARTR & survival & Control & lppd & -24.19 & -23.06 & -1.13 &  \\ 
	14 & ARTR & survival & Control & MSE & 0.07 & 0.07 & 0.00 &  \\ 
	15 & ARTR & survival & Drought & lppd & -5.48 & -5.34 & -0.13 &  \\ 
	16 & ARTR & survival & Drought & MSE & 0.04 & 0.04 & -0.00 & *** \\ 
	17 & ARTR & survival & Irrigation & lppd & -7.88 & -8.01 & 0.13 & *** \\ 
	18 & ARTR & survival & Irrigation & MSE & 0.06 & 0.06 & -0.00 & *** \\ 
	19 & HECO & growth & Control & lppd & -377.77 & -369.46 & -8.31 &  \\ 
	20 & HECO & growth & Control & MSE & 1.11 & 1.09 & 0.02 &  \\ 
	21 & HECO & growth & Drought & lppd & -8.74 & -10.97 & 2.23 & *** \\ 
	22 & HECO & growth & Drought & MSE & 3.12 & 4.35 & -1.23 & *** \\ 
	23 & HECO & growth & Irrigation & lppd & -89.03 & -73.93 & -15.10 &  \\ 
	24 & HECO & growth & Irrigation & MSE & 2.02 & 1.47 & 0.55 &  \\ 
	25 & HECO & recruitment & Control & lppd & -93.35 & -93.47 & 0.12 & *** \\ 
	26 & HECO & recruitment & Control & MSE & 613.41 & 499.11 & 114.29 &  \\ 
	27 & HECO & recruitment & Drought & lppd & -25.39 & -27.56 & 2.17 & *** \\ 
	28 & HECO & recruitment & Drought & MSE & 1.94 & 2.33 & -0.38 & *** \\ 
	29 & HECO & recruitment & Irrigation & lppd & -30.69 & -30.56 & -0.12 &  \\ 
	30 & HECO & recruitment & Irrigation & MSE & 5.21 & 7.62 & -2.41 & *** \\ 
	31 & HECO & survival & Control & lppd & -124.59 & -112.70 & -11.90 &  \\ 
	32 & HECO & survival & Control & MSE & 0.12 & 0.11 & 0.01 &  \\ 
	33 & HECO & survival & Drought & lppd & -17.89 & -20.82 & 2.93 & *** \\ 
	34 & HECO & survival & Drought & MSE & 0.22 & 0.26 & -0.05 & *** \\ 
	35 & HECO & survival & Irrigation & lppd & -16.51 & -13.54 & -2.97 &  \\ 
	36 & HECO & survival & Irrigation & MSE & 0.09 & 0.07 & 0.02 &  \\ 
	37 & POSE & growth & Control & lppd & -1117.20 & -1117.29 & 0.08 & *** \\ 
	38 & POSE & growth & Control & MSE & 1.50 & 1.51 & -0.00 & *** \\ 
	39 & POSE & growth & Drought & lppd & -254.17 & -257.32 & 3.16 & *** \\ 
	40 & POSE & growth & Drought & MSE & 2.66 & 2.69 & -0.03 & *** \\ 
	41 & POSE & growth & Irrigation & lppd & -452.34 & -456.95 & 4.61 & *** \\ 
	42 & POSE & growth & Irrigation & MSE & 1.87 & 1.90 & -0.03 & *** \\ 
	43 & POSE & recruitment & Control & lppd & -127.64 & -128.73 & 1.09 & *** \\ 
	44 & POSE & recruitment & Control & MSE & 35.20 & 44.82 & -9.61 & *** \\ 
	45 & POSE & recruitment & Drought & lppd & -60.34 & -63.15 & 2.80 & *** \\ 
	46 & POSE & recruitment & Drought & MSE & 23.60 & 33.88 & -10.28 & *** \\ 
	47 & POSE & recruitment & Irrigation & lppd & -72.06 & -65.58 & -6.48 &  \\ 
	48 & POSE & recruitment & Irrigation & MSE & 85.04 & 27.63 & 57.41 &  \\ 
	49 & POSE & survival & Control & lppd & -366.49 & -380.49 & 14.00 & *** \\ 
	50 & POSE & survival & Control & MSE & 0.12 & 0.12 & -0.00 & *** \\ 
	51 & POSE & survival & Drought & lppd & -175.29 & -187.88 & 12.59 & *** \\ 
	52 & POSE & survival & Drought & MSE & 0.21 & 0.22 & -0.01 & *** \\ 
	53 & POSE & survival & Irrigation & lppd & -156.28 & -150.23 & -6.05 &  \\ 
	54 & POSE & survival & Irrigation & MSE & 0.13 & 0.13 & 0.00 &  \\ 
	55 & PSSP & growth & Control & lppd & -627.07 & -625.63 & -1.44 &  \\ 
	56 & PSSP & growth & Control & MSE & 1.38 & 1.36 & 0.02 &  \\ 
	57 & PSSP & growth & Drought & lppd & -292.90 & -297.61 & 4.71 & *** \\ 
	58 & PSSP & growth & Drought & MSE & 1.82 & 1.85 & -0.03 & *** \\ 
	59 & PSSP & growth & Irrigation & lppd & -312.96 & -314.67 & 1.71 & *** \\ 
	60 & PSSP & growth & Irrigation & MSE & 1.51 & 1.51 & -0.00 & *** \\ 
	61 & PSSP & recruitment & Control & lppd & -123.81 & -125.23 & 1.42 & *** \\ 
	62 & PSSP & recruitment & Control & MSE & 30.59 & 34.79 & -4.20 & *** \\ 
	63 & PSSP & recruitment & Drought & lppd & -87.34 & -86.58 & -0.77 &  \\ 
	64 & PSSP & recruitment & Drought & MSE & 222.16 & 54.58 & 167.58 &  \\ 
	65 & PSSP & recruitment & Irrigation & lppd & -60.18 & -62.12 & 1.94 & *** \\ 
	66 & PSSP & recruitment & Irrigation & MSE & 20.90 & 44.57 & -23.67 & *** \\ 
	67 & PSSP & survival & Control & lppd & -148.69 & -140.15 & -8.54 &  \\ 
	68 & PSSP & survival & Control & MSE & 0.10 & 0.09 & 0.01 &  \\ 
	69 & PSSP & survival & Drought & lppd & -83.26 & -78.09 & -5.17 &  \\ 
	70 & PSSP & survival & Drought & MSE & 0.11 & 0.11 & 0.01 &  \\ 
	71 & PSSP & survival & Irrigation & lppd & -100.52 & -89.02 & -11.50 &  \\ 
	72 & PSSP & survival & Irrigation & MSE & 0.13 & 0.11 & 0.01 &  \\ 
	\hline
	\caption{Comparison of model predictions from climate model and year effects model for each species and vital rate and treatment.  Two prediction scores are reported, MSE and lppd. Lower MSE indicates improved predictions whereas higher lppd indicates improved predictions.  Instances where the climate model outperformed the random year effects model are marked with "***" in the last column. ARTR = \textit{A. tripartita}, HECO = \textit{H. comata}, POSE = \textit{P. secunda}, PSSP = \textit{P. spicata}.} 
	\label{table:treatmentPreds}
\end{longtable}


\clearpage
\newpage
\clearpage
\newpage
\section{Additional Figures} 

\begin{figure}[!htbp]
	\centering
	\includegraphics[width=1\textwidth]{pred_v_obs_treatment_ARTR_growth}
	\caption{Comparison of treatment effects predicted and observed for \textit{A. tripartita} growth.  Upper figure shows each parameter estimate separately while the lower figure shows the effect of treatment as a function of plant size.  The observed treatment effects come from the treatment model fitted to all the data including the five years of the experiment.  The predicted parameters come from the climate model fitted to all years of observational data but do not include the five years of the experiment (2011 to 2016). We assessed a parameter as significant when the 95\% Bayesian credible intervals did not overlap zero.}
	\label{fig:parPredARTRGrowth}
\end{figure}

\begin{figure}[!htbp]
	\centering
	\includegraphics[width=1\textwidth]{pred_v_obs_treatment_HECO_growth}
	\caption{Comparison of treatment effects predicted and observed for \textit{H. comata} growth.  The observed treatment effects come from the treatment model fitted to all the data including the five years of the experiment.  The predicted parameters come from the climate model fitted to all years of observational data but do not include the five years of the experiment (2011 to 2016). We assessed a parameter as significant when the 95\% Bayesian credible intervals did not overlap zero.}
	\label{fig:parPredHECOGrowth}
\end{figure}

\begin{figure}[!htbp]
	\centering
	\includegraphics[width=1\textwidth]{pred_v_obs_treatment_POSE_growth}
	\caption{Comparison of treatment effects predicted and observed for \textit{P. secunda} growth.  Upper figure shows each parameter estimate separately while the lower figure shows the effect of treatment as a function of plant size.  The observed treatment effects come from the treatment model fitted to all the data including the five years of the experiment.  The predicted parameters come from the climate model fitted to all years of observational data but do not include the five years of the experiment (2011 to 2016). We assessed a parameter as significant when the 95\% Bayesian credible intervals did not overlap zero.}
	\label{fig:parPredPOSEGrowth}
\end{figure}


\begin{figure}[!htbp]
	\centering
	\includegraphics[width=1\textwidth]{pred_v_obs_treatment_PSSP_growth}
	\caption{Comparison of treatment effects predicted and observed for \textit{P. spicata} growth.  The observed treatment effects come from the treatment model fitted to all the data including the five years of the experiment.  The predicted parameters come from the climate model fitted to all years of observational data but do not include the five years of the experiment (2011 to 2016). We assessed a parameter as significant when the 95\% Bayesian credible intervals did not overlap zero.}
	\label{fig:parPredPSSPGrowth}
\end{figure}


\begin{figure}[!htbp]
	\centering
	\includegraphics[width=1\textwidth]{pred_v_obs_treatment_ARTR_survival}
	\caption{Comparison of treatment effects predicted and observed for \textit{A. survival} survival.  The observed treatment effects come from the treatment model fitted to all the data including the five years of the experiment.  The predicted parameters come from the climate model fitted to all years of observational data but do not include the five years of the experiment (2011 to 2016). We assessed a parameter as significant when the 95\% Bayesian credible intervals did not overlap zero.}
	\label{fig:parPredARTRSurvival}
\end{figure}

\begin{figure}[!htbp]
	\centering
	\includegraphics[width=1\textwidth]{pred_v_obs_treatment_HECO_survival}
	\caption{Comparison of treatment effects predicted and observed for \textit{H. comata} survival.  The observed treatment effects come from the treatment model fitted to all the data including the five years of the experiment.  The predicted parameters come from the climate model fitted to all years of observational data but do not include the five years of the experiment (2011 to 2016). We assessed a parameter as significant when the 95\% Bayesian credible intervals did not overlap zero.}
	\label{fig:parPredHECOSurvival}
\end{figure}


\begin{figure}[!htbp]
	\centering
	\includegraphics[width=1\textwidth]{pred_v_obs_treatment_POSE_survival}
	\caption{Comparison of treatment effects predicted and observed for \textit{P. secunda} survival.  Upper figure shows each parameter estimate separately while the lower figure shows the effect of treatment as a function of plant size.  The observed treatment effects come from the treatment model fitted to all the data including the five years of the experiment.  The predicted parameters come from the climate model fitted to all years of observational data but do not include the five years of the experiment (2011 to 2016). We assessed a parameter as significant when the 95\% Bayesian credible intervals did not overlap zero.}
	\label{fig:parPredPOSESurvival}
\end{figure}

\begin{figure}[!htbp]
	\centering
	\includegraphics[width=1\textwidth]{pred_v_obs_treatment_PSSP_survival}
	\caption{Comparison of treatment effects predicted and observed for \textit{P. spicata} survival.  The observed treatment effects come from the treatment model fitted to all the data including the five years of the experiment.  The predicted parameters come from the climate model fitted to all years of observational data but do not include the five years of the experiment (2011 to 2016). We assessed a parameter as significant when the 95\% Bayesian credible intervals did not overlap zero.}
	\label{fig:parPredPSSPSurvival}
\end{figure}



\end{document}

