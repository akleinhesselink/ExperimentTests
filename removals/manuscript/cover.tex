\documentclass[11pt]{letter}
%\renewcommand{\baselinestretch}{2}
%\setlength{\headheight}{2in}
\setlength{\topmargin}{0in}
\setlength{\headsep}{0in}
\setlength{\oddsidemargin}{0in}
\setlength{\evensidemargin}{0in}
\setlength{\textheight}{8.5in}
\setlength{\textwidth}{6.5in}   
%\setlength{\textheight}{6.5in}   

\usepackage{graphicx}



\begin{document}

Peter Adler\\
 Department of Wildland Resources and the Ecology Center\\
 5230 Old Main\\
 Utah State University, Logan, UT 84322\\
 Tel. 801-910-3816 \quad Fax:  435-797-3796\\
 email: peter.adler@usu.edu
 
 \medskip

Dear Editors,

Please consider our manuscript, ``Weak interspecific interactions in a sagebrush steppe: evidence from observations, models, and experiments''
for publication as an Article in \textit{Ecology}.

Interspecific competition is arguably the primary obsession of plant community ecologists. However, in a series of papers based on long-term demographic data and multispecies population models for semiarid plant communities (Adler et al. 2010, Adler et al. 2012, Chu and Adler 2015), my collaborators and I found surprisingly weak interspecific interactions. These results imply little competitive release; removal of a dominant species should have little impact on the abundances of remaining species.

To test this prediction, we conducted a five year removal experiment in a sagebrush steppe plant community. We found that models based on long-term observational data adequately 
predicted vital rate and population-level responses to removals for three of our four study species. For the fourth species, the model based on observational data underestimated population growth in the removal treatment, but additional analyses suggest that this mismatch may not actually reflect competitive release. 
 
Overall, the results of our experiment increase our confidence that the structure and dynamics of this plant community are determined more by variation in intraspecific interactions than by strong interspecific competition for resources. We expect this conclusion may surprise a broad audience of plant ecologists accustomed to studies emphasizing the importance of interspecific competition. Our results also support the use of observational data to build multispecies population models and explore community dynamics.

For the convenience of reviewers, we have placed the Figures and Tables in the main text, but the final 
version (if the paper is accepted) will be structured as specified in the Author Guidelines. All data and R code needed
to reproduce the analysis are currently 
available on Github (see Methods), and will be deposited in a permanent archive upon acceptance of the manuscript.

None of this material has been published or is under consideration elsewhere. I am the corresponding author and my contact
information is listed above. All authors have seen and approved the manuscript. 

We look forward to hearing from you. 

Sincerely,

Peter Adler and coauthors

\textbf{References}

Adler, P.B., Dalgleish, H.J. and Ellner, S.P. 2012. Forecasting plant community impacts of climate variability and change: when do competitive interactions matter? Journal of Ecology 100: 478-487.

Adler, P.B., Ellner, S.P. and Levine, J.M. 2010. Coexistence of perennial plants: an embarrassment of niches. Ecology Letters 13: 1019-1029.

Chu, C. and Adler, P.B. 2015. Large niche differences emerge at the recruitment stage to stabilize grassland coexistence. Ecological Monographs 85: 373-392.




\end{document}

